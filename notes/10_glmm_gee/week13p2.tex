% Options for packages loaded elsewhere
\PassOptionsToPackage{unicode}{hyperref}
\PassOptionsToPackage{hyphens}{url}
\PassOptionsToPackage{dvipsnames,svgnames,x11names}{xcolor}
%
\documentclass[
  ignorenonframetext,
]{beamer}
\usepackage{pgfpages}
\setbeamertemplate{caption}[numbered]
\setbeamertemplate{caption label separator}{: }
\setbeamercolor{caption name}{fg=normal text.fg}
\beamertemplatenavigationsymbolsempty
% Prevent slide breaks in the middle of a paragraph
\widowpenalties 1 10000
\raggedbottom
\setbeamertemplate{part page}{
  \centering
  \begin{beamercolorbox}[sep=16pt,center]{part title}
    \usebeamerfont{part title}\insertpart\par
  \end{beamercolorbox}
}
\setbeamertemplate{section page}{
  \centering
  \begin{beamercolorbox}[sep=12pt,center]{part title}
    \usebeamerfont{section title}\insertsection\par
  \end{beamercolorbox}
}
\setbeamertemplate{subsection page}{
  \centering
  \begin{beamercolorbox}[sep=8pt,center]{part title}
    \usebeamerfont{subsection title}\insertsubsection\par
  \end{beamercolorbox}
}
\AtBeginPart{
  \frame{\partpage}
}
\AtBeginSection{
  \ifbibliography
  \else
    \frame{\sectionpage}
  \fi
}
\AtBeginSubsection{
  \frame{\subsectionpage}
}
\usepackage{amsmath,amssymb}
\usepackage{lmodern}
\usepackage{iftex}
\ifPDFTeX
  \usepackage[T1]{fontenc}
  \usepackage[utf8]{inputenc}
  \usepackage{textcomp} % provide euro and other symbols
\else % if luatex or xetex
  \usepackage{unicode-math}
  \defaultfontfeatures{Scale=MatchLowercase}
  \defaultfontfeatures[\rmfamily]{Ligatures=TeX,Scale=1}
\fi
% Use upquote if available, for straight quotes in verbatim environments
\IfFileExists{upquote.sty}{\usepackage{upquote}}{}
\IfFileExists{microtype.sty}{% use microtype if available
  \usepackage[]{microtype}
  \UseMicrotypeSet[protrusion]{basicmath} % disable protrusion for tt fonts
}{}
\makeatletter
\@ifundefined{KOMAClassName}{% if non-KOMA class
  \IfFileExists{parskip.sty}{%
    \usepackage{parskip}
  }{% else
    \setlength{\parindent}{0pt}
    \setlength{\parskip}{6pt plus 2pt minus 1pt}}
}{% if KOMA class
  \KOMAoptions{parskip=half}}
\makeatother
\usepackage{xcolor}
\newif\ifbibliography
\usepackage{color}
\usepackage{fancyvrb}
\newcommand{\VerbBar}{|}
\newcommand{\VERB}{\Verb[commandchars=\\\{\}]}
\DefineVerbatimEnvironment{Highlighting}{Verbatim}{commandchars=\\\{\}}
% Add ',fontsize=\small' for more characters per line
\usepackage{framed}
\definecolor{shadecolor}{RGB}{248,248,248}
\newenvironment{Shaded}{\begin{snugshade}}{\end{snugshade}}
\newcommand{\AlertTok}[1]{\textcolor[rgb]{0.94,0.16,0.16}{#1}}
\newcommand{\AnnotationTok}[1]{\textcolor[rgb]{0.56,0.35,0.01}{\textbf{\textit{#1}}}}
\newcommand{\AttributeTok}[1]{\textcolor[rgb]{0.77,0.63,0.00}{#1}}
\newcommand{\BaseNTok}[1]{\textcolor[rgb]{0.00,0.00,0.81}{#1}}
\newcommand{\BuiltInTok}[1]{#1}
\newcommand{\CharTok}[1]{\textcolor[rgb]{0.31,0.60,0.02}{#1}}
\newcommand{\CommentTok}[1]{\textcolor[rgb]{0.56,0.35,0.01}{\textit{#1}}}
\newcommand{\CommentVarTok}[1]{\textcolor[rgb]{0.56,0.35,0.01}{\textbf{\textit{#1}}}}
\newcommand{\ConstantTok}[1]{\textcolor[rgb]{0.00,0.00,0.00}{#1}}
\newcommand{\ControlFlowTok}[1]{\textcolor[rgb]{0.13,0.29,0.53}{\textbf{#1}}}
\newcommand{\DataTypeTok}[1]{\textcolor[rgb]{0.13,0.29,0.53}{#1}}
\newcommand{\DecValTok}[1]{\textcolor[rgb]{0.00,0.00,0.81}{#1}}
\newcommand{\DocumentationTok}[1]{\textcolor[rgb]{0.56,0.35,0.01}{\textbf{\textit{#1}}}}
\newcommand{\ErrorTok}[1]{\textcolor[rgb]{0.64,0.00,0.00}{\textbf{#1}}}
\newcommand{\ExtensionTok}[1]{#1}
\newcommand{\FloatTok}[1]{\textcolor[rgb]{0.00,0.00,0.81}{#1}}
\newcommand{\FunctionTok}[1]{\textcolor[rgb]{0.00,0.00,0.00}{#1}}
\newcommand{\ImportTok}[1]{#1}
\newcommand{\InformationTok}[1]{\textcolor[rgb]{0.56,0.35,0.01}{\textbf{\textit{#1}}}}
\newcommand{\KeywordTok}[1]{\textcolor[rgb]{0.13,0.29,0.53}{\textbf{#1}}}
\newcommand{\NormalTok}[1]{#1}
\newcommand{\OperatorTok}[1]{\textcolor[rgb]{0.81,0.36,0.00}{\textbf{#1}}}
\newcommand{\OtherTok}[1]{\textcolor[rgb]{0.56,0.35,0.01}{#1}}
\newcommand{\PreprocessorTok}[1]{\textcolor[rgb]{0.56,0.35,0.01}{\textit{#1}}}
\newcommand{\RegionMarkerTok}[1]{#1}
\newcommand{\SpecialCharTok}[1]{\textcolor[rgb]{0.00,0.00,0.00}{#1}}
\newcommand{\SpecialStringTok}[1]{\textcolor[rgb]{0.31,0.60,0.02}{#1}}
\newcommand{\StringTok}[1]{\textcolor[rgb]{0.31,0.60,0.02}{#1}}
\newcommand{\VariableTok}[1]{\textcolor[rgb]{0.00,0.00,0.00}{#1}}
\newcommand{\VerbatimStringTok}[1]{\textcolor[rgb]{0.31,0.60,0.02}{#1}}
\newcommand{\WarningTok}[1]{\textcolor[rgb]{0.56,0.35,0.01}{\textbf{\textit{#1}}}}
\usepackage{graphicx}
\makeatletter
\def\maxwidth{\ifdim\Gin@nat@width>\linewidth\linewidth\else\Gin@nat@width\fi}
\def\maxheight{\ifdim\Gin@nat@height>\textheight\textheight\else\Gin@nat@height\fi}
\makeatother
% Scale images if necessary, so that they will not overflow the page
% margins by default, and it is still possible to overwrite the defaults
% using explicit options in \includegraphics[width, height, ...]{}
\setkeys{Gin}{width=\maxwidth,height=\maxheight,keepaspectratio}
% Set default figure placement to htbp
\makeatletter
\def\fps@figure{htbp}
\makeatother
\setlength{\emergencystretch}{3em} % prevent overfull lines
\providecommand{\tightlist}{%
  \setlength{\itemsep}{0pt}\setlength{\parskip}{0pt}}
\setcounter{secnumdepth}{-\maxdimen} % remove section numbering
\usepackage{graphicx}
\usepackage{bm}
\definecolor{foreground}{RGB}{255,255,255}
\definecolor{background}{RGB}{34,28,54}
\definecolor{title}{RGB}{105,165,255}
\definecolor{gray}{RGB}{175,175,175}
\definecolor{lightgray}{RGB}{225,225,225}
\definecolor{subtitle}{RGB}{232,234,255}
\definecolor{hilight}{RGB}{112,224,255}
\definecolor{vhilight}{RGB}{255,111,207}
\setbeamertemplate{footline}[page number]
\ifLuaTeX
  \usepackage{selnolig}  % disable illegal ligatures
\fi
\IfFileExists{bookmark.sty}{\usepackage{bookmark}}{\usepackage{hyperref}}
\IfFileExists{xurl.sty}{\usepackage{xurl}}{} % add URL line breaks if available
\urlstyle{same} % disable monospaced font for URLs
\hypersetup{
  pdftitle={STAT 528 - Advanced Regression Analysis II},
  pdfauthor={GLMM and GEE},
  colorlinks=true,
  linkcolor={Maroon},
  filecolor={Maroon},
  citecolor={Blue},
  urlcolor={blue},
  pdfcreator={LaTeX via pandoc}}

\title{STAT 528 - Advanced Regression Analysis II}
\author{GLMM and GEE}
\date{}
\institute{Daniel J. Eck\\
Department of Statistics\\
University of Illinois}

\begin{document}
\frame{\titlepage}

\begin{frame}
\newcommand{\R}{\mathbb{R}}
\end{frame}

\begin{frame}{Learning Objectives Today}
\protect\hypertarget{learning-objectives-today}{}
\begin{itemize}
\tightlist
\item
  GLMM examples
\item
  GEE theory
\item
  GEE examples
\end{itemize}
\end{frame}

\begin{frame}[fragile]{}
\protect\hypertarget{section}{}
We load in necessary packages.

\vspace{12pt}

\begin{Shaded}
\begin{Highlighting}[]
\FunctionTok{library}\NormalTok{(faraway)}
\FunctionTok{library}\NormalTok{(tidyverse)}
\FunctionTok{library}\NormalTok{(ggplot2)}
\FunctionTok{library}\NormalTok{(MASS)}
\FunctionTok{library}\NormalTok{(lme4)}
\FunctionTok{library}\NormalTok{(INLA)}
\FunctionTok{library}\NormalTok{(glmm)}
\FunctionTok{library}\NormalTok{(parallel)}
\end{Highlighting}
\end{Shaded}
\end{frame}

\begin{frame}[fragile]{}
\protect\hypertarget{section-1}{}
In this example, we have data from a clinical trial of 59 epileptics.

For a baseline, patients were observed for 8 weeks and the number of
seizures recorded. The patients were then randomized to treatment by the
drug Progabide (31 patients) or to the placebo group (28 patients).

They were observed for four 2-week periods and the number of seizures
recorded. We are interested in determining whether Progabide reduces the
rate of seizures.

We first perform some data manipulations and then look at the first few
observations:

\vspace{12pt}
\tiny

\begin{Shaded}
\begin{Highlighting}[]
\FunctionTok{data}\NormalTok{(epilepsy, }\AttributeTok{package=}\StringTok{"faraway"}\NormalTok{)}
\NormalTok{epilepsy}\SpecialCharTok{$}\NormalTok{period }\OtherTok{\textless{}{-}} \FunctionTok{rep}\NormalTok{(}\DecValTok{0}\SpecialCharTok{:}\DecValTok{4}\NormalTok{, }\DecValTok{59}\NormalTok{)}
\NormalTok{epilepsy}\SpecialCharTok{$}\NormalTok{drug }\OtherTok{\textless{}{-}} \FunctionTok{factor}\NormalTok{(}\FunctionTok{c}\NormalTok{(}\StringTok{"placebo"}\NormalTok{,}\StringTok{"treatment"}\NormalTok{)[epilepsy}\SpecialCharTok{$}\NormalTok{treat}\SpecialCharTok{+}\DecValTok{1}\NormalTok{])}
\NormalTok{epilepsy}\SpecialCharTok{$}\NormalTok{phase }\OtherTok{\textless{}{-}} \FunctionTok{factor}\NormalTok{(}\FunctionTok{c}\NormalTok{(}\StringTok{"baseline"}\NormalTok{,}\StringTok{"experiment"}\NormalTok{)[epilepsy}\SpecialCharTok{$}\NormalTok{expind }\SpecialCharTok{+}\DecValTok{1}\NormalTok{])}
\NormalTok{epilepsy }\SpecialCharTok{\%\textgreater{}\%} \FunctionTok{filter}\NormalTok{(id }\SpecialCharTok{\textless{}} \FloatTok{2.5}\NormalTok{) }\SpecialCharTok{\%\textgreater{}\%} \FunctionTok{head}\NormalTok{(}\DecValTok{3}\NormalTok{)}
\end{Highlighting}
\end{Shaded}

\begin{verbatim}
##   seizures id treat expind timeadj age period    drug      phase
## 1       11  1     0      0       8  31      0 placebo   baseline
## 2        5  1     0      1       2  31      1 placebo experiment
## 3        3  1     0      1       2  31      2 placebo experiment
\end{verbatim}
\end{frame}

\begin{frame}[fragile]{}
\protect\hypertarget{section-2}{}
The variables are \texttt{expind} variable indicates the baseline phase
by 0 and the treatment phase by 1. The length of these time phases is
recorded in the \texttt{timeadj} variable.

Three new convenience variables are created: \texttt{period}, denoting
the 2- or 8- week periods, \texttt{drug} recording the type of treatment
in nonnumeric form and \texttt{phase} indicating the phase of the
experiment.

We now compute the mean number of seizures per week broken down by the
treatment and baseline vs.~experimental period.

\vspace{12pt}
\tiny

\begin{Shaded}
\begin{Highlighting}[]
\NormalTok{epilepsy }\SpecialCharTok{\%\textgreater{}\%} 
  \FunctionTok{group\_by}\NormalTok{(drug, phase) }\SpecialCharTok{\%\textgreater{}\%} 
  \FunctionTok{summarise}\NormalTok{(}\AttributeTok{rate=}\FunctionTok{mean}\NormalTok{(seizures}\SpecialCharTok{/}\NormalTok{timeadj)) }\SpecialCharTok{\%\textgreater{}\%}
\FunctionTok{xtabs}\NormalTok{(}\AttributeTok{formula=}\NormalTok{rate }\SpecialCharTok{\textasciitilde{}}\NormalTok{ phase }\SpecialCharTok{+}\NormalTok{ drug)}
\end{Highlighting}
\end{Shaded}

\begin{verbatim}
##             drug
## phase         placebo treatment
##   baseline   3.848214  3.955645
##   experiment 4.303571  3.983871
\end{verbatim}
\end{frame}

\begin{frame}{}
\protect\hypertarget{section-3}{}
We see that the rate of seizures in the treatment group actually
increases during the period in which the drug was taken. The rate of
seizures increases even more in the placebo group.

Perhaps some other factor is causing the rate of seizures to increase
during the treatment period and the drug is actually having a beneficial
effect.

Now we make some plots to show the difference between the treatment and
the control. The first plot shows the difference between the two groups
during the experimental period only:
\end{frame}

\begin{frame}{}
\protect\hypertarget{section-4}{}
\includegraphics{week13p2_files/figure-beamer/unnamed-chunk-4-1.pdf}
\end{frame}

\begin{frame}{}
\protect\hypertarget{section-5}{}
We now compare the average seizure rate to the baseline for the two
groups. The square-root transform is used to stabilize the variance;
this is often used with count data.

\vspace{12pt}

\includegraphics{week13p2_files/figure-beamer/unnamed-chunk-5-1.pdf}
\end{frame}

\begin{frame}[fragile]{}
\protect\hypertarget{section-6}{}
A treatment effect, if one exists, is not readily apparent. Now we fit
GLMM models. Patient \#49 is unusual because of the high rate of
seizures observed. We exclude it:

\vspace{12pt}
\small

\begin{Shaded}
\begin{Highlighting}[]
\NormalTok{epilo }\OtherTok{\textless{}{-}} \FunctionTok{filter}\NormalTok{(epilepsy, id }\SpecialCharTok{!=} \DecValTok{49}\NormalTok{)}
\end{Highlighting}
\end{Shaded}

\vspace{12pt}
\normalsize

Excluding a case should not be taken lightly. For projects where the
analyst works with producers of the data, it will be possible to discuss
substantive reasons for excluding cases.

It is worth starting with a GLM even though the model is not correct due
to the grouping of the observations. We must use an offset to allow for
the difference in lengths in the baseline and treatment periods.
\end{frame}

\begin{frame}[fragile]{}
\protect\hypertarget{section-7}{}
\tiny

\begin{Shaded}
\begin{Highlighting}[]
\NormalTok{modglm }\OtherTok{\textless{}{-}} \FunctionTok{glm}\NormalTok{(seizures }\SpecialCharTok{\textasciitilde{}}\FunctionTok{offset}\NormalTok{(}\FunctionTok{log}\NormalTok{(timeadj)) }\SpecialCharTok{+}\NormalTok{ expind }\SpecialCharTok{+}\NormalTok{ treat }\SpecialCharTok{+} 
  \FunctionTok{I}\NormalTok{(expind}\SpecialCharTok{*}\NormalTok{treat), }\AttributeTok{family=}\NormalTok{poisson, }\AttributeTok{data=}\NormalTok{epilo)}
\FunctionTok{summary}\NormalTok{(modglm)}
\end{Highlighting}
\end{Shaded}

\begin{verbatim}
## 
## Call:
## glm(formula = seizures ~ offset(log(timeadj)) + expind + treat + 
##     I(expind * treat), family = poisson, data = epilo)
## 
## Deviance Residuals: 
##     Min       1Q   Median       3Q      Max  
## -5.4725  -2.3605  -1.0290   0.9001  14.0104  
## 
## Coefficients:
##                   Estimate Std. Error z value Pr(>|z|)    
## (Intercept)        1.34761    0.03406  39.566  < 2e-16 ***
## expind             0.11184    0.04688   2.386    0.017 *  
## treat             -0.10682    0.04863  -2.197    0.028 *  
## I(expind * treat) -0.30238    0.06971  -4.338 1.44e-05 ***
## ---
## Signif. codes:  0 '***' 0.001 '**' 0.01 '*' 0.05 '.' 0.1 ' ' 1
## 
## (Dispersion parameter for poisson family taken to be 1)
## 
##     Null deviance: 2485.1  on 289  degrees of freedom
## Residual deviance: 2411.5  on 286  degrees of freedom
## AIC: 3449.7
## 
## Number of Fisher Scoring iterations: 5
\end{verbatim}
\end{frame}

\begin{frame}{}
\protect\hypertarget{section-8}{}
The interaction term is the primary parameter of interest. All the
subjects were untreated in the baseline. This means that the main effect
for treatment does not properly measure the response to treatment
because it includes the baseline period.

As we have observed already, we suspect the response may have been
different during the baseline time and the active period of the
experiment. The interaction term represents the effect of the treatment
during the baseline period after adjustment. In the output above we see
that this interaction seems highly significant and negative (which is
good since we want to reduce seizures).

But this inference is suspect because we have made no allowance for the
correlated responses within individuals. The p-value is far smaller than
it should be.
\end{frame}

\begin{frame}[fragile]{PQL methods}
\protect\hypertarget{pql-methods}{}
\tiny

\begin{Shaded}
\begin{Highlighting}[]
\NormalTok{modpql }\OtherTok{\textless{}{-}} \FunctionTok{glmmPQL}\NormalTok{(seizures }\SpecialCharTok{\textasciitilde{}}\FunctionTok{offset}\NormalTok{(}\FunctionTok{log}\NormalTok{(timeadj)) }\SpecialCharTok{+}\NormalTok{ expind }\SpecialCharTok{+}\NormalTok{ treat }\SpecialCharTok{+} 
  \FunctionTok{I}\NormalTok{(expind}\SpecialCharTok{*}\NormalTok{treat), }\AttributeTok{random =} \SpecialCharTok{\textasciitilde{}}\DecValTok{1}\SpecialCharTok{|}\NormalTok{id, }\AttributeTok{family=}\NormalTok{poisson, }\AttributeTok{data=}\NormalTok{epilo)}
\FunctionTok{summary}\NormalTok{(modpql)}
\end{Highlighting}
\end{Shaded}

\begin{verbatim}
## Linear mixed-effects model fit by maximum likelihood
##   Data: epilo 
##   AIC BIC logLik
##    NA  NA     NA
## 
## Random effects:
##  Formula: ~1 | id
##         (Intercept) Residual
## StdDev:   0.6820012 1.605385
## 
## Variance function:
##  Structure: fixed weights
##  Formula: ~invwt 
## Fixed effects:  seizures ~ offset(log(timeadj)) + expind + treat + I(expind *      treat) 
##                        Value  Std.Error  DF   t-value p-value
## (Intercept)        1.0761832 0.09990514 230 10.772050  0.0000
## expind             0.1125119 0.07412152 230  1.517939  0.1304
## I(expind * treat) -0.3037615 0.10819095 230 -2.807642  0.0054
##  Correlation: 
##                   (Intr) expind
## expind            -0.198       
## I(expind * treat) -0.014 -0.656
## 
## Standardized Within-Group Residuals:
##        Min         Q1        Med         Q3        Max 
## -2.2934834 -0.5649468 -0.1492931  0.3224895  6.3123337 
## 
## Number of Observations: 290
## Number of Groups: 58
\end{verbatim}
\end{frame}

\begin{frame}{}
\protect\hypertarget{section-9}{}
The parameter estimates from the PQL fit are comparable to the GLM fit.
However, the standard errors are larger in the PQL fit as might be
expected given that the correlated responses have been allowed for.

As with the binary response example, we still have some doubts about the
accuracy of the inference. This is a particular concern when some count
responses are small.
\end{frame}

\begin{frame}[fragile]{Numerical integration}
\protect\hypertarget{numerical-integration}{}
Numerical quadrature can also be used. We use Gauss-Hermite in
preference to Laplace as the model random effect structure is simple and
so the computation is fast even though we have used the most expensive
\texttt{nAGQ=25} setting. \vspace{12pt} \tiny

\begin{Shaded}
\begin{Highlighting}[]
\NormalTok{modgh }\OtherTok{\textless{}{-}} \FunctionTok{glmer}\NormalTok{(seizures }\SpecialCharTok{\textasciitilde{}}\FunctionTok{offset}\NormalTok{(}\FunctionTok{log}\NormalTok{(timeadj)) }\SpecialCharTok{+}\NormalTok{ expind }\SpecialCharTok{+}\NormalTok{ treat }\SpecialCharTok{+} 
  \FunctionTok{I}\NormalTok{(expind}\SpecialCharTok{*}\NormalTok{treat)}\SpecialCharTok{+}\NormalTok{ (}\DecValTok{1}\SpecialCharTok{|}\NormalTok{id), }\AttributeTok{nAGQ=}\DecValTok{25}\NormalTok{, }\AttributeTok{family=}\NormalTok{poisson, }\AttributeTok{data=}\NormalTok{epilo)}
\end{Highlighting}
\end{Shaded}
\end{frame}

\begin{frame}[fragile]{}
\protect\hypertarget{section-10}{}
\tiny

\begin{Shaded}
\begin{Highlighting}[]
\FunctionTok{summary}\NormalTok{(modgh)}
\end{Highlighting}
\end{Shaded}

\begin{verbatim}
## Generalized linear mixed model fit by maximum likelihood (Adaptive
##   Gauss-Hermite Quadrature, nAGQ = 25) [glmerMod]
##  Family: poisson  ( log )
## Formula: seizures ~ offset(log(timeadj)) + expind + treat + I(expind *  
##     treat) + (1 | id)
##    Data: epilo
## 
##      AIC      BIC   logLik deviance df.resid 
##    877.7    896.1   -433.9    867.7      285 
## 
## Scaled residuals: 
##     Min      1Q  Median      3Q     Max 
## -3.8724 -0.8482 -0.1722  0.5697  9.8941 
## 
## Random effects:
##  Groups Name        Variance Std.Dev.
##  id     (Intercept) 0.515    0.7176  
## Number of obs: 290, groups:  id, 58
## 
## Fixed effects:
##                    Estimate Std. Error z value Pr(>|z|)    
## (Intercept)        1.035998   0.141256   7.334 2.23e-13 ***
## expind             0.111838   0.046877   2.386    0.017 *  
## treat             -0.008152   0.196525  -0.041    0.967    
## I(expind * treat) -0.302387   0.069714  -4.338 1.44e-05 ***
## ---
## Signif. codes:  0 '***' 0.001 '**' 0.01 '*' 0.05 '.' 0.1 ' ' 1
## 
## Correlation of Fixed Effects:
##             (Intr) expind treat 
## expind      -0.175              
## treat       -0.718  0.126       
## I(xpnd*trt)  0.118 -0.672 -0.173
\end{verbatim}
\end{frame}

\begin{frame}[fragile]{}
\protect\hypertarget{section-11}{}
We see that the interaction effect is significant. Notice that the
estimate of this effect has been quite consistent over all the
estimation methods so we draw some confidence from this. We have

\vspace{12pt}

\begin{Shaded}
\begin{Highlighting}[]
\FunctionTok{exp}\NormalTok{(}\SpecialCharTok{{-}}\FloatTok{0.302}\NormalTok{)}
\end{Highlighting}
\end{Shaded}

\begin{verbatim}
## [1] 0.7393381
\end{verbatim}

\vspace{12pt}

So the drug is estimated to reduce the rate of seizures by about 26\%.
However, the subject SD is more than twice the drug effect of -0.3 at
0.718. This indicates that the expected improvement in the drug is
substantially less than the variation between individuals.
\end{frame}

\begin{frame}{}
\protect\hypertarget{section-12}{}
Interpretation of the main effect terms is problematic in the presence
of an interaction. For example, the treatment effect reported here
represents the predicted difference in the response during the baseline
period (i.e., \texttt{expind=0}).

Since none of the subjects are treated during the baseline period, we
are reassured to see that this effect is not significant.

However, this does illustrate the danger in naively presuming that this
is the treatment effect.
\end{frame}

\begin{frame}[fragile]{Bayesian methods}
\protect\hypertarget{bayesian-methods}{}
We can also take a Bayesian approach using \texttt{INLA}.

\vspace{12pt}
\tiny

\begin{Shaded}
\begin{Highlighting}[]
\NormalTok{formula }\OtherTok{\textless{}{-}}\NormalTok{ seizures }\SpecialCharTok{\textasciitilde{}} \FunctionTok{offset}\NormalTok{(}\FunctionTok{log}\NormalTok{(timeadj)) }\SpecialCharTok{+}\NormalTok{ expind }\SpecialCharTok{+}\NormalTok{ treat }\SpecialCharTok{+} 
  \FunctionTok{I}\NormalTok{(expind}\SpecialCharTok{*}\NormalTok{treat) }\SpecialCharTok{+} \FunctionTok{f}\NormalTok{(id,}\AttributeTok{model=}\StringTok{"iid"}\NormalTok{)}
\NormalTok{result }\OtherTok{\textless{}{-}} \FunctionTok{inla}\NormalTok{(formula, }\AttributeTok{family=}\StringTok{"poisson"}\NormalTok{, }\AttributeTok{data =}\NormalTok{ epilo)}
\end{Highlighting}
\end{Shaded}

\vspace{12pt}
\normalsize

We obtain a summary of the posteriors as:

\vspace{12pt}
\tiny

\begin{Shaded}
\begin{Highlighting}[]
\NormalTok{sigmaalpha }\OtherTok{\textless{}{-}} \FunctionTok{inla.tmarginal}\NormalTok{(}\ControlFlowTok{function}\NormalTok{(x) }\DecValTok{1}\SpecialCharTok{/}\FunctionTok{sqrt}\NormalTok{(x), }
\NormalTok{  result}\SpecialCharTok{$}\NormalTok{marginals.hyperpar}\SpecialCharTok{$}\StringTok{"Precision for id"}\NormalTok{)}
\NormalTok{restab }\OtherTok{\textless{}{-}} \FunctionTok{sapply}\NormalTok{(result}\SpecialCharTok{$}\NormalTok{marginals.fixed, }
  \ControlFlowTok{function}\NormalTok{(x) }\FunctionTok{inla.zmarginal}\NormalTok{(x, }\AttributeTok{silent=}\ConstantTok{TRUE}\NormalTok{))}
\NormalTok{restab }\OtherTok{\textless{}{-}} \FunctionTok{cbind}\NormalTok{(restab, }
  \FunctionTok{inla.zmarginal}\NormalTok{(sigmaalpha, }\AttributeTok{silent=}\ConstantTok{TRUE}\NormalTok{))}
\FunctionTok{colnames}\NormalTok{(restab) }\OtherTok{=} \FunctionTok{c}\NormalTok{(}\StringTok{"mu"}\NormalTok{,}\StringTok{"expind"}\NormalTok{,}\StringTok{"treat"}\NormalTok{,}
  \StringTok{"interaction"}\NormalTok{,}\StringTok{"alpha"}\NormalTok{)}
\FunctionTok{data.frame}\NormalTok{(restab)}
\end{Highlighting}
\end{Shaded}

\begin{verbatim}
##                   mu     expind        treat interaction      alpha
## mean        1.036115  0.1118965  -0.00825552  -0.3025702  0.7254792
## sd         0.1421983 0.04685179    0.1978247  0.06967626 0.07183047
## quant0.025 0.7550146 0.01995068   -0.3978168  -0.4393086    0.59977
## quant0.25  0.9410767 0.08018646    -0.140758  -0.3497282  0.6746726
## quant0.5    1.036183  0.1117989 -0.008691486  -0.3027154   0.719905
## quant0.75   1.130938  0.1434113     0.123375  -0.2557025  0.7702153
## quant0.975  1.314419  0.2036471     0.380427  -0.1661221  0.8818008
\end{verbatim}

\vspace{12pt}
\normalsize

We see that the results are similar to those obtained previously. We
observe that the 95\% credible interval for the interaction is
(-0.44,-0.17) so we are sure that this parameter differs from zero. We
compute similar plots as we did in the binary response example.
\end{frame}

\begin{frame}{}
\protect\hypertarget{section-13}{}
\includegraphics{week13p2_files/figure-beamer/unnamed-chunk-14-1.pdf}
\end{frame}

\begin{frame}[fragile]{Monte Carlo likelihood approximation}
\protect\hypertarget{monte-carlo-likelihood-approximation}{}
We can use the \texttt{glmm} package to implement the MCLA approach to
fitting GLMM models with Poisson responses.

\vspace{12pt}
\tiny

\begin{Shaded}
\begin{Highlighting}[]
\NormalTok{epilo}\SpecialCharTok{$}\NormalTok{idF }\OtherTok{\textless{}{-}} \FunctionTok{as.factor}\NormalTok{(epilo}\SpecialCharTok{$}\NormalTok{id)}
\NormalTok{epilo}\SpecialCharTok{$}\NormalTok{seizures }\OtherTok{\textless{}{-}} \FunctionTok{as.integer}\NormalTok{(epilo}\SpecialCharTok{$}\NormalTok{seizures)}
\FunctionTok{set.seed}\NormalTok{(}\DecValTok{13}\NormalTok{)}
\NormalTok{clust }\OtherTok{\textless{}{-}} \FunctionTok{makeCluster}\NormalTok{(}\DecValTok{8}\NormalTok{)}
\FunctionTok{system.time}\NormalTok{(m1 }\OtherTok{\textless{}{-}} \FunctionTok{glmm}\NormalTok{(seizures }\SpecialCharTok{\textasciitilde{}} 
\NormalTok{  expind }\SpecialCharTok{+}\NormalTok{ treat }\SpecialCharTok{+} \FunctionTok{I}\NormalTok{(expind}\SpecialCharTok{*}\NormalTok{treat), }\AttributeTok{random =} \FunctionTok{list}\NormalTok{(}\SpecialCharTok{\textasciitilde{}}\DecValTok{0}\SpecialCharTok{+}\NormalTok{idF), }
  \AttributeTok{family.glmm =}\NormalTok{ poisson.glmm, }\AttributeTok{m =} \FloatTok{7e4}\NormalTok{, }
  \AttributeTok{varcomps.names =} \FunctionTok{c}\NormalTok{(}\StringTok{"idF"}\NormalTok{), }\AttributeTok{cluster =}\NormalTok{ clust, }\AttributeTok{data=}\NormalTok{epilo))}
\end{Highlighting}
\end{Shaded}

\begin{verbatim}
##    user  system elapsed 
##   6.186   0.632  68.164
\end{verbatim}
\end{frame}

\begin{frame}[fragile]{}
\protect\hypertarget{section-14}{}
We obtain summary information. However, the fit is buggy. The Monte
Carlo standard error is not returned and the summary table estimates are
not depicted.

\vspace{12pt}
\tiny

\begin{Shaded}
\begin{Highlighting}[]
\FunctionTok{summary}\NormalTok{(m1)}
\end{Highlighting}
\end{Shaded}

\begin{verbatim}
## 
## Call:
## glmm(fixed = seizures ~ expind + treat + I(expind * treat), random = list(~0 + 
##     idF), varcomps.names = c("idF"), data = epilo, family.glmm = poisson.glmm, 
##     m = 70000, cluster = clust)
## 
## 
## Link is: "log"
## 
## Fixed Effects:
##                   Estimate Std. Error z value Pr(>|z|)    
## (Intercept)              0          0  31.010  < 2e-16 ***
## expind                   0          0 -27.187  < 2e-16 ***
## treat                    0          0   0.018    0.986    
## I(expind * treat)        0          0  -4.338 1.44e-05 ***
## ---
## Signif. codes:  0 '***' 0.001 '**' 0.01 '*' 0.05 '.' 0.1 ' ' 1
## 
## 
## Variance Components for Random Effects (P-values are one-tailed):
##     Estimate Std. Error z value Pr(>|z|)/2    
## idF        0          0   5.098   1.72e-07 ***
## ---
## Signif. codes:  0 '***' 0.001 '**' 0.01 '*' 0.05 '.' 0.1 ' ' 1
\end{verbatim}
\end{frame}

\begin{frame}[fragile]{}
\protect\hypertarget{section-15}{}
\tiny

\begin{Shaded}
\begin{Highlighting}[]
\CommentTok{\# Monte Carlo standard errors}
\NormalTok{mcse\_glmm }\OtherTok{\textless{}{-}} \FunctionTok{mcse}\NormalTok{(m1)}
\NormalTok{mcse\_glmm}
\end{Highlighting}
\end{Shaded}

\begin{verbatim}
##       (Intercept)            expind             treat I(expind * treat) 
##               NaN               NaN               NaN               NaN 
##               idF 
##               NaN
\end{verbatim}
\end{frame}

\begin{frame}[fragile]{}
\protect\hypertarget{section-16}{}
That being said, we can obtain estimates of fixed effects and their
standard errors from objects in the \texttt{glmm} object.

\vspace{12pt}
\tiny

\begin{Shaded}
\begin{Highlighting}[]
\CommentTok{\# standard errors}
\NormalTok{se\_glmm }\OtherTok{\textless{}{-}} \FunctionTok{se}\NormalTok{(m1)}

\CommentTok{\# table for fixed effects}
\NormalTok{tab }\OtherTok{\textless{}{-}} \FunctionTok{cbind}\NormalTok{(m1}\SpecialCharTok{$}\NormalTok{beta, se\_glmm[}\SpecialCharTok{{-}}\DecValTok{5}\NormalTok{], m1}\SpecialCharTok{$}\NormalTok{beta}\SpecialCharTok{/}\NormalTok{se\_glmm[}\SpecialCharTok{{-}}\DecValTok{5}\NormalTok{])}
\FunctionTok{colnames}\NormalTok{(tab) }\OtherTok{\textless{}{-}} \FunctionTok{c}\NormalTok{(}\StringTok{"Estimate"}\NormalTok{, }\StringTok{"Std. Error"}\NormalTok{, }\StringTok{"z value"}\NormalTok{)}
\FunctionTok{round}\NormalTok{(tab, }\DecValTok{3}\NormalTok{)}
\end{Highlighting}
\end{Shaded}

\begin{verbatim}
##                   Estimate Std. Error z value
## (Intercept)          3.106      0.100  31.010
## expind              -1.274      0.047 -27.187
## treat                0.003      0.185   0.018
## I(expind * treat)   -0.302      0.070  -4.338
\end{verbatim}
\end{frame}

\begin{frame}[fragile]{}
\protect\hypertarget{section-17}{}
We can also obtain estimates of random effect parameters and their
standard errors from objects in the \texttt{glmm} object.

\vspace{12pt}
\tiny

\begin{Shaded}
\begin{Highlighting}[]
\FunctionTok{c}\NormalTok{(m1}\SpecialCharTok{$}\NormalTok{nu, se\_glmm[}\DecValTok{5}\NormalTok{])}
\end{Highlighting}
\end{Shaded}

\begin{verbatim}
##       idF       idF 
## 0.5152511 0.1010674
\end{verbatim}

\vspace{12pt}
\normalsize

Parameter estimates are similar to the other fitting techniques which
instills confidence.
\end{frame}

\end{document}
